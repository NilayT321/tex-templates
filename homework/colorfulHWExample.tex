\documentclass[mathshortcuts, colorful]{homework}

\class{MATH 670 -- Measure Theory}
\assignmentNo{Homework 1}
\dueDate{October 28, 2025}

\studentName{Ash Ketchup}
\instructorName{Professor Oak}
\sectionNo{01}

\begin{document}
		\maketitle 

		\noindent The \verb|\maketitle| command has been redefined to produce the assignment header above. By default, this header includes the class name, assignment number, due date, as well as details about the class. \bigbreak 

		\noindent We have loaded the \texttt{mathshortcuts} option to load in a set of sensible math shortcuts. Refer to the source in \texttt{homework.cls} for the full set. If you have your own set of macros, just load the class without this option to use them. \bigbreak

		\noindent This assignment has been the loaded with the \texttt{colorful} option to add some color to your boring mathematical life. This generates a simplistic yet stylish colored box to contain the problem. The colored box is typed using the \texttt{cproblem} environment; this way, you can still use the plain and boring problem environment if you find a need to. Note that the counters for each type of box are separate at this time.
		
		\begin{cproblem}
				State the definition of a $\sigma$-algebra on a set $X$ and of a measure on $X$.
		\end{cproblem}
		\begin{soln}
				A $\sigma$-algebra on $X$ is a collection of subsets $\calS \subseteq 2^X$ such that 
				\begin{enumerate}
						\item $\emptyset \in \calS$. 
						\item $\calS$ is closed under countable unions. 
						\item $\calS$ is closed under complements.
				\end{enumerate}
				If $X$ is a set and $\calS$ is a $\sigma$-algebra on $X$, then a \textbf{measure} on $X$ is a set function $\mu : \calS \to [0, +\infty]$ satisfying: 
				\begin{enumerate}
						\item $\mu(\emptyset) = 0$ 
						\item $\mu \left( \bigcup_{k=1}^{\infty} E_k \right) = \sum_{k=1}^{\infty} \mu(E_k)$ if the collection $\{E_k\}$ consists of disjoint sets.
				\end{enumerate}
		\end{soln}

		The default color is \texttt{cyan!10} (note that the \texttt{xcolor} package is being used). You can change the color by passing in an optional argument to the \texttt{cproblem} environment like this: 
		\begin{verbatim}
						\begin{cproblem}[red!30] 
								Your problem here...
						\end{cproblem}
		\end{verbatim}

		\begin{cproblem}[red!30] 
				Give an example of a non-Lebesgue measurable subset of $\R$ (this problem is red since it is harder than the last problem).
		\end{cproblem}
		\begin{soln}
				We can construct a Vitali set.....
		\end{soln}
\end{document}
